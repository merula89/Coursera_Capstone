\documentclass[10pt,a4paper]{article}
\usepackage[latin1]{inputenc}
\usepackage[T1]{fontenc}
\usepackage{amsmath}
\usepackage{amsfonts}
\usepackage{amssymb}
\usepackage{graphicx}
\usepackage{hyperref}

\title{Coursera Capstone Proposal - Suicide Rates}
\author{Manuel Ramsaier \\
	RWU University, Germany \\
}

\date{\today}
% Hint: \title{what ever}, \author{who care} and \date{when ever} could stand 
% before or after the \begin{document} command 
% BUT the \maketitle command MUST come AFTER the \begin{document} command! 
\begin{document}
	
	\maketitle
	
	
	\begin{abstract}
		Later there will be an abstract here
	\end{abstract}
	
	\section{Introduction}
	In the previous examples, we looked at comparing neighbourhoods from Toronto or New York. Big cities with huge skyscrapers and god knows, those are beautiful places to live. But what about the districts in southern Germany, where I come from. I want to apply the lessons learned in a more rural context. Which districts are similar, which districts have most inhabitants and so on. 
	
	As there is a need for a direct problem statement, I will try to find a good place to build a fitness centre. 
	
	\section{Problem Statement}

	Imagine, you want to invest in a fitness centre but you are not sure, where to start it. There are lots of different variables 

	
	\section{Data used} \label{documentclasses}
	
	The suicide data is taken from \href{http://opendatalab.de/projects/geojson-utilities/}}. There I selected the districts around me and exported the json file. It will give access to a lot of information useful as a starting point. \\

	First I will import the geojson file into a dataframe and extract the features I am interested in, namely the descriptive name and location of the district. This information is then used to get venues via foursquare. \\ 
	

	

%	\begin{itemize}
%		\item article
%		\item book 
%		\item report 
%		\item letter 
%	\end{itemize}
	
	
%	\begin{enumerate}
%		\item article
%		\item book 
%		\item report 
%		\item letter 
%	\end{enumerate}
%	
%	\begin{description}
%		\item[article\label{article}]{Article is \ldots}
%		\item[book\label{book}]{The book class \ldots}
%		\item[report\label{report}]{Report gives you \ldots}
%		\item[letter\label{letter}]{If you want to write a letter.}
%	\end{description}
%	
%	
%	\section{Conclusions}\label{conclusions}
%	There is no longer \LaTeX{} example which was written by \cite{doe}.
%	
%	
%	\begin{thebibliography}{9}
%		\bibitem[Doe]{doe} \emph{First and last \LaTeX{} example.},
%		John Doe 50 B.C. 
%	\end{thebibliography}
	
\end{document}